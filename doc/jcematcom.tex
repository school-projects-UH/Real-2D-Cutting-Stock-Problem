%===================================================================================
% JORNADA CIENTÍFICA ESTUDIANTIL - MATCOM, UH
%===================================================================================
% Esta plantilla ha sido diseñada para ser usada en los artículos de la
% Jornada Científica Estudiantil, MatCom.
%
% Por favor, siga las instrucciones de esta plantilla y rellene en las secciones
% correspondientes.
%
% NOTA: Necesitará el archivo 'jcematcom.sty' en la misma carpeta donde esté este
%       archivo para poder utilizar esta plantila.
%===================================================================================



%===================================================================================
% PREÁMBULO
%-----------------------------------------------------------------------------------
\documentclass[a4paper,10pt,twocolumn]{article}

%===================================================================================
% Paquetes
%-----------------------------------------------------------------------------------
\usepackage{amsmath}
\usepackage{amsfonts}
\usepackage{amssymb}
\usepackage{jcematcom}
\usepackage[utf8]{inputenc}
\usepackage{listings}
\usepackage[pdftex]{hyperref}
%-----------------------------------------------------------------------------------
% Configuración
%-----------------------------------------------------------------------------------
\hypersetup{colorlinks,%
	    citecolor=black,%
	    filecolor=black,%
	    linkcolor=black,%
	    urlcolor=blue}

%===================================================================================



%===================================================================================
% Presentacion
%-----------------------------------------------------------------------------------
% Título
%-----------------------------------------------------------------------------------
\title{Título}

%-----------------------------------------------------------------------------------
% Autores
%-----------------------------------------------------------------------------------
\author{\\
\name Dalianys Pérez Pereira \email \href{mailto:a.uno@lab.matcom.uh.cu}{a.uno@lab.matcom.uh.cu}
	\\ \addr Grupo C411 \AND
\name Dayany Alfaro González \email \href{mailto:a.dos@lab.matcom.uh.cu}{a.dos@lab.matcom.uh.cu}
  \\ \addr Grupo C411 \AND
\name Gilberto González Rodríguez \email \href{mailto:a.dos@lab.matcom.uh.cu}{a.dos@lab.matcom.uh.cu}
\\ \addr Grupo C411 \AND
\name Antonio Jesús Otaño Barrera \email \href{mailto:a.dos@lab.matcom.uh.cu}{a.dos@lab.matcom.uh.cu}
\\ \addr Grupo C411}

%-----------------------------------------------------------------------------------
% Tutores
%-----------------------------------------------------------------------------------
%\tutors{\\
%Dr. Tutor Uno, \emph{Centro} \\
%Lic. Tutor Dos, \emph{Centro}}

%-----------------------------------------------------------------------------------
% Headings
%-----------------------------------------------------------------------------------
%\jcematcomheading{\the\year}{1-\pageref{end}}{A. Uno, A. Dos}

%-----------------------------------------------------------------------------------
%\ShortHeadings{Autores}
%===================================================================================



%===================================================================================
% DOCUMENTO
%-----------------------------------------------------------------------------------
\begin{document}

%-----------------------------------------------------------------------------------
% NO BORRAR ESTA LINEA!
%-----------------------------------------------------------------------------------
\twocolumn[
%-----------------------------------------------------------------------------------

\maketitle

%===================================================================================
% Resumen y Abstract
%-----------------------------------------------------------------------------------
\selectlanguage{spanish} % Para producir el documento en Español

%-----------------------------------------------------------------------------------
% Resumen en Español
%-----------------------------------------------------------------------------------
\begin{abstract}

	El Resumen en Español debe constar de $100$ a $200$ palabras y presentar de forma
	clara y concisa el contenido fundamental del artículo.

\end{abstract}

%-----------------------------------------------------------------------------------
% English Abstract
%-----------------------------------------------------------------------------------
\vspace{0.5cm}

\begin{enabstract}

  The English Abstract must have have $100$ to $200$ words, and present in a clear
  and concise form the essentials of the article content.

\end{enabstract}

%-----------------------------------------------------------------------------------
% Palabras clave
%-----------------------------------------------------------------------------------
\begin{keywords}
	Separadas,
	Por,
	Comas.
\end{keywords}

%-----------------------------------------------------------------------------------
% Temas
%-----------------------------------------------------------------------------------
\begin{topics}
	Tema, Subtema.
\end{topics}


%-----------------------------------------------------------------------------------
% NO BORRAR ESTAS LINEAS!
%-----------------------------------------------------------------------------------
\vspace{0.8cm}
]
%-----------------------------------------------------------------------------------


%===================================================================================

%===================================================================================
% Introducción
%-----------------------------------------------------------------------------------
\section{Introducción}\label{sec:intro}
%-----------------------------------------------------------------------------------
El problema de patrones de corte de piezas rectangulares pertenece a
la familia de problemas de corte y empaquetamiento y sus
aplicaciones se pueden observar en industrias de perfiles
metálicos, corte de maderas, papel, plástico o vidrio en
donde los componentes rectangulares tienen que ser
cortados de grandes hojas de material. Para estas industrias es de gran importancia realizar
este proceso de corte de una manera eficiente buscando
minimizar el desperdicio y los demás costos asociados
al proceso, teniendo en cuenta las restricciones técnicas
y de demanda. 



El problema de patrones de corte es un problema de gran
complejidad tanto por las características y variables
que involucra como por las técnicas que se utilizan
para abordarlo, es una temática en constante evolución
y muchos investigadores han desarrollado diversos
modelos para resolverlo. El interés en este problema
puede ser sustentado por su aplicación práctica y el
reto que representa pues, en general,
es computacionalmente difícil de resolver ya que es un
problema de tipo NP-completo, dado que los patrones de
empaquetamiento incrementan exponencialmente con el
número de rectángulos que deben ser empaquetados. 
 
Este trabajo tiene como objetivo resolver un problema de corte en dos dimensiones, asociado a la industria del papel, presente en una empresa ubicada en la provinicia Pinar del Río, Cuba. Se propone el  diseño e implementación de un algoritmo que permita determinar qué patrones de corte deben usarse para cortar un conjunto de hojas de forma que se satisfaga una demanda (de hojas más pequeñas) solicitada por el usuario de forma que el desperdicio resultante de los cortes sea el menor posible. En este caso se permite la rotación de las piezas a colocar y se requiere que los cortes sean de tipo guillotina, es decir, que  el corte vaya de un extremo a otro del rectángulo original. 

[Texto explicando lo que aborda cada sección]

%===================================================================================

\section{Antecedentes y Enfoques de Solución}
El Problema de patrones de corte (CSP, por sus siglas
en inglés) fue formulado por primera vez en 1939
por el economista ruso Kantorovich.

Han surgido numerosas investigaciones que abordan
diferentes problemas según el tipo de dimensión (1D y
2D) y desde diversos enfoques tales como los métodos
exactos, heurísticos y meta heurísticos, pero aún no
existe un método global establecido para dar solución
a este tipo de problemas, debido a la complejidad
asociada. 

\subsection{Programación Lineal Entera}

Casi todos los procedimientos basados en la
programación lineal para resolver el problema de
patrones de corte se remontan a Gilmore y Gomory,
\cite{1}, para lo cual, proponen la relajación de la restricción
de integridad para la solución de problemas de
programación lineal logrando minimizar el desperdicio
a través de la generación de columnas evitando el
conocimiento explícito o enumeración de todos los
patrones desde el principio, ya que bajo este esquema
las columnas (patrones) son generadas cuando se
requieran \cite{2}. La idea consiste en utilizar el método
simplex revisado para resolver el problema de la entrada
del patrón de corte siguiente a la base mediante la
resolución de un problema de la mochila asociado. Este
método es denominado en la literatura como \textit{delayed
column generation technique}, y permite resolver este
tipo de problemas en un tiempo computacional mucho
menor \cite{3}.


\subsection{Procedimientos Heurísticos Secuenciales}
Los procedimientos heurísticos secuenciales pertenecen
a la clase de heurísticas de búsqueda local. La solución
se construye mediante la generación de patrones uno
a uno hasta que todos los requerimientos de demanda
se hayan satisfecho, donde los patrones inicialmente
seleccionados deben tener un nivel de desperdicio
bajo, un nivel de utilización alto y dejar una serie de requerimientos para poder combinar bien los patrones
futuros, evitando así incurrir posteriormente en
desperdicios excesivos \cite{3}. La ventaja principal de este método es que puede controlar otros factores aparte del desperdicio y elimina
el problema del redondeo al trabajar sólo con valores
enteros.

\subsection{Procedimientos Heurísticos Híbridos }
Este procedimiento consiste en combinar los dos
procedimientos descritos anteriormente, de tal forma
que se utilice el procedimiento heurístico secuencial
para generar una solución, la cual es guardada y
utilizada como base inicial en el procedimiento de
programación lineal. Posteriormente, el desperdicio es
reducido si es posible a través iteraciones adicionales,
tal y como lo realiza \cite{41}. 

Independientemente de la forma como
se combinen estos dos métodos, lo más importante
del éxito de la unión entre el procedimiento heurístico
secuencial y el redondeo de problemas de programación
lineal es la selección del criterio apropiado para resolver
el problema \cite{29}.

\subsection{ Metaheurísticas }
Ante el problema que presenta la búsqueda local y
las heurísticas constructivas de quedar atrapadas en
óptimos locales, surgen las metaheurísticas a mediados
de 1970 pues tienen la capacidad de guiar la búsqueda
local para que se escape de los óptimos locales. Muchos
de estos algoritmos se han utilizado para resolver
el problema de patrones de corte, entre los cuales
se destaca \textit{Tabu Search} (TS), \textit{Greedy Randomized
Adaptive Search Procedure} (GRASP) \cite{42}, Algoritmos
genéticos \cite{43,46} y \textit{Ant Colony Optimization} (ACO) \cite{47,48}, entre otros algoritmos evolucionarios \cite{51}.

%===================================================================================
% Desarrollo
%-----------------------------------------------------------------------------------
\section{Definición del Problema}

\section{Propuesta de Solución}

\section{Desarrollo}\label{sec:dev}
%-----------------------------------------------------------------------------------
  En esta sección (o secciones) incluya el contenido fundamental del artículo.
  No es necesario tener una sección nombrada \emph{Desarrollo}, por el contrario,
  nombre las secciones según el contenido que tratan.

%-----------------------------------------------------------------------------------
	\subsection{Organización del Documento}\label{sub:results}
%-----------------------------------------------------------------------------------
		Puede agregar secciones y subsecciones según sea necesario para organizar
		de manera más coherente su artículo. Tenga en cuenta que un documento más
		plano es más fácil de navegar y entender, pero las subsecciones relacionadas
		deberían estar agrupadas en una sección común.

		Los nombres de las secciones deben ir en mayúsculas, excepto para las
		preposiciones, conjunciones, y otros vocablos auxiliares.

		Empiece un nuevo párrafo cada vez que vaya a comenzar una idea nueva.

%-----------------------------------------------------------------------------------
	\subsection{Listas y Descripciones}\label{sub:lists}
%-----------------------------------------------------------------------------------
		Para producir listas enumeradas, use el siguiente estilo:

%-----------------------------------------------------------------------------------
		\begin{enumerate}
			\item Primer Elemento
			\item Segundo Elemento
			%
			\begin {enumerate}
				\item {Segundo Elemento - Subitem Uno}
				\item {Segundo Elemento - Subitem Dos}
			\end {enumerate}
			%
		\end{enumerate}

%-----------------------------------------------------------------------------------
		Para producir descripciones, use el siguiente estilo:

%-----------------------------------------------------------------------------------
		\begin{description}
			\item [Primer Elemento] con su respectiva descripción.
			\item [Segundo Elemento] también con su respectiva descripción.
		\end{description}

%-----------------------------------------------------------------------------------
	\subsection{Figuras}\label{sub:figures}
%-----------------------------------------------------------------------------------
		Para producir cuerpos flotantes (figuras ó tablas), asegúrese de numerar
		y etiquetar correctamente cada figura. Las referencias a las figuras deben
		estar también correctamente etiquetadas. Por ejemplo, en la Fig. \ref{fig:ex}
		se muestra\ldots.

		\begin{figure}[htb]%
		\begin{center}
			\emph{Aquí va el contenido de la figura \ldots}
		\end{center}
		\caption{Figura de ejemplo \label{fig:ex}}%
		\end{figure}

%-----------------------------------------------------------------------------------
	\subsection{Código Fuente}\label{sub:listings}
%-----------------------------------------------------------------------------------
		Para producir código fuente, envuélvalo en una figura flotante y
		etiquételo correctamente. Por ejemplo, en la Fig. \ref{fig:code}
		se muestra un código bastante conocido\ldots.

		% Configuración de Listings
		\lstset{keywordstyle=\color{blue}, basicstyle=\small}

		\begin{figure}[htb]%
			\begin{lstlisting}[language=c]%

    int main(int argc, char** argv)
    {
        // Imprimiendo "Hola Mundo".
        printf("Hello, World");
    }

			\end{lstlisting}
		\caption{Código fuente de ejemplo.\label{fig:code}}
		\end{figure}

%-----------------------------------------------------------------------------------
	\subsection{Referencias}
%-----------------------------------------------------------------------------------
  	Las referencias deben estar agrupadas en una sección al final del artículo,
  	y las citas numeradas correctamente, por ejemplo \cite{knuth} ó \cite{goedel}.
  	Incluya toda la información importante de cada referencia, incluídos autor,
  	título, y notas de la edición. En caso de citar sitios web, además
  	de la URL, incluya la fecha en que fue consultado, como en \cite{wiki}.

%===================================================================================



%===================================================================================
% Conclusiones
%-----------------------------------------------------------------------------------
\section{Conclusiones}\label{sec:conc}

  En esta sección puede incluir las conclusiones de su investigación y las ideas
  sobre la continuidad del trabajo, en el caso que aplique.

%===================================================================================



%===================================================================================
% Recomendaciones
%-----------------------------------------------------------------------------------
\section{Recomendaciones}\label{sec:rec}

  En esta sección puede incluir recomendaciones sobre posibles formas de continuar
  la investigación u otros temas relacionados.

%===================================================================================



%===================================================================================
% Bibliografía
%-----------------------------------------------------------------------------------
\begin{thebibliography}{99}
%-----------------------------------------------------------------------------------
	\bibitem{1} P. Gilmore, and R. Gomory, “A linear programming
	approach to the Cutting Stock Problem-Part II,”
	Operations Research, 11(6), 863-888, 1963.
	
	\bibitem{2} H. Hideki, and M.J. Pinto, “An integrated cutting
	stock and sequencing problem,” European Journal of
	Operational Research (183), 1353–1370, 2007.
	
	\bibitem{3}J. Karelahti, “Solving the cutting stock problem in
	the steel industry”. Department of Engineering Physics
	and Mathematics. Helsinki University of Technology,
	2-5, 2002.
	
	\bibitem{41}  Cui, Y.-P., Tang, T.-B. Parallelized sequential value correction procedure for the one-dimensional cutting
	stock problem with multiple stock lengths. Engineering
	Optimization, 46 (10), 1352-1368, 2014.
	
	\bibitem{29} R. Haessler, and P. Sweeney, “Cutting stock
	problems and solution procedures,” European Journal
	of Operational Research, 54, 141-150, 1991.
	
	\bibitem{42} MirHassani, S.A., Jalaeian Bashirzadeh, A. A
	GRASP meta-heuristic for two-dimensional irregular
	cutting stock problem. International Journal of
	Advanced Manufacturing Technology, 81 (1-4), 455-
	464, 2105.
	
	\bibitem{43}  Wenshu, L., Dan, M., Jinzhuo, W. Study on cutting
	stock optimization for decayed wood board based
	on genetic algorithm. Open Automation and Control
	Systems Journal, 7 (1), 284-289, 2015.
	
	\bibitem{46}  Lu, H.-C.a , Huang, Y.-H.b. An efficient genetic
	algorithm with a corner space algorithm for a cutting
	stock problem in the TFT-LCD industry. European
	Journal of Operational Research, 246 (1), 51-65, 2015.
	
	\bibitem{47}Lu, Q., Zhou, X. GPU parallel ant colony algorithm
	for the dynamic one-dimensional cutting stock problem
	based on the on-line detection. Yi Qi Yi Biao Xue Bao/
	Chinese Journal of Scientific Instrument, 36 (8), pp.
	1774-1782, 2015.
	
	\bibitem{48}  Díaz, D., Valledor, P., Areces, P., Rodil, J., Suárez, M.
	An ACO Algorithm to Solve an Extended Cutting Stock
	Problem for Scrap Minimization in a Bar Mill. Lecture
	Notes in Computer Science (including subseries Lecture
	Notes in Artificial Intelligence and Lecture Notes in
	Bioinformatics), 8667, 13-24, 2014.
	
	\bibitem{51} Ben Lagha, G.a , Dahmani, N.b , Krichen, S.a.
	Particle swarm optimization approach for resolving the
	cutting stock problem. 2014 International Conference
	on Advanced Logistics and Transport, 2014.
	
	
	
	\bibitem{knuth} Donald E. Knuth. \emph{The Art of Computer Programming}.
		Volume 1: Fundamental Algorithms (3rd~edition), 1997.
		Addison-Wesley Professional.

	\bibitem{goedel} Kurt Göedel. \emph{Über formal unentscheidbare Sätze der
		Principia Mathematica und verwandter Systeme, I}.
		Monatshefte für Mathematik und Physik 38.

	\bibitem{wiki} Wikipedia. URL: \href{http://en.wikipedia.org}
	  {http://en.wikipedia.org}.
		Consultado en \today.

%-----------------------------------------------------------------------------------
\end{thebibliography}

%-----------------------------------------------------------------------------------

\label{end}

\end{document}

%===================================================================================
